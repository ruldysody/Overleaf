\documentclass[dvipdfmx, 12pt]{jarticle}

\usepackage{template}

\usepackage{amsmath,amssymb,amsfonts,amsthm}
\usepackage[dvipdfmx]{graphicx} %画像の貼り付け
\usepackage{here} %絶対位置
\usepackage{subcaption} %\subcaption用
\usepackage{url}
\usepackage{enumerate}
\usepackage{float}
\allowdisplaybreaks[1] %(節,式番号) 形式で式の番号を表します
\eqswitchon 
%定理環境の定義
\newtheorem{theorem}{定理}[section]
\newtheorem{lemma}{補題}[section]
\newtheorem{definition}{定義}[section]


\header{2023年度}
%タイトル
\title{フロー需要に伴う施設利用者に着目した\\
\vspace{4mm}通過量モデル}
%著者名
\author{金子大悟}
%学籍番号
\studentId{62005693}
%指導教員
\supervisor{教授}{田中健一}
%修了年月
\completedDay{2024年}{3月}
%専攻名
\institute{慶應義塾大学理工学部}{管理工学科}

\begin{document}
\maketitle


%謝辞
\begin{acknowledgments}

研究の遂行,本論文の作成にあたり,基礎的なのモデルの例を示し,丁寧に解説してくださるなど研究テーマ決定から本論文の完成までの道のりにおいて要所要所で有益な助言をいただきました,田中教授に深謝致します.研究テーマがなかなか決まらず,ORの知識も全くなかった私に時に厳しく助言をいただきました.本論文が作成できたのは間違いなく田中教授のおかげです.また,田中研究室の皆様には,本研究の遂行にあたり多大なご助言,ご協力頂きました.ここに感謝の意を表します.

\sign{2024年1月31日}{慶應義塾大学理工学部}{管理工学科}{田中研究室}
\end{acknowledgments}


\pagenumbering{roman}
\tableofcontents
\newpage
\pagenumbering{arabic}

%%%%%%%%%%%%%%%%%%%%%%%%%%%%%%%%%%%%%%%%%%%%%%%%%%%%%%%%%%%%%%%%%%%%%%%%%%%%%%%%%%%%%%%%%%%%%%%%%%%%%%%%%%%%%%%%%%%%%%%%%%%%%%%%%%%%%%%%
\section{はじめに}

\end{document}